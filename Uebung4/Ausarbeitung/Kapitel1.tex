\chapter*{Ausarbeitung}

\section{Gezeitenpotential}

Gegeben sei der Punkt $P$ mit den sphärischen Koordinaten ($\lambda = 8.33^{\circ},~ \varphi = 48.14^{\circ},~r = 6366837~m$). Im Folgenden wird berechnet: 

\begin{enumerate}[a)]
\item Berechnung der Zeitreihe der Koeffizienten $v_{2,m}^{tid}$ 

Die Zeitreihe der Koeffizienten $v_{2,m}^{tid}$ vom Grad 2 des vom Mond erzeugten Gezeitenpotentials werden nun berechnet. Dabei werden explizit die Werte vom 15. Januar angegeben. Um die Zeitreihe zu berechnen wird folgende Formel herangezogen: 

\begin{gather}
V^{tid}(\lambda, \phi, r) = \sum_{l=2}^{L} \sum_{m=-l}^{l} \left(\dfrac{r}{R}\right)^l v_{l,m}^{tid} \overline{Y}_{l,m}(\lambda,\phi) 
\label{vtid}
\end{gather} 
\begin{gather}
v_{l,m}^{tid} = \dfrac{GM_{Moon}}{r_{Moon}} \dfrac{1}{2l+1} \left(\dfrac{R}{r_{Moon}}\right)^l \overline{Y}_{l,m} (\lambda_{Moon},\phi_{Moon})
\end{gather}
\begin{center}
$= \begin{bmatrix}
-0.5885 &  0.4409 &  -0.8039  & -0.0927 & -1.3381
\end{bmatrix}$
\end{center}

\item Berechnung des Gezeitenpotentials $v_{tid}$ 

Hier wird das vom Mond erzeugte Gezeitenpotential $v_{tid}$ berechnet. Anschließend werden die Zahlenwerte für den 1. bis 5. Januar angegeben. Bereits oben erklärt Formel \ref{vtid} die Berechnung des Gezeitenpotentials. Diese wird angewandt und es ergeben sich folgende Ergebnisse für die Tage des ersten bis fünften Januars. 

\begin{gather*}
V_{tid} = \begin{bmatrix}
-1.4564&-1.3540&-1.2456&-1.1379&-1.0388
\end{bmatrix}~[m^2/s^2]
\end{gather*}

\begin{figure}[H]
\centering
\includegraphics[scale=0.6]{gezpot.png}
\caption{Gezeitenpotential im Berechnungspunkt}
\label{gezpot}
\end{figure}

Abbildung \ref{gezpot} zeigt das Gezeitenpotential im Berechnungspunkt für das gesamte Zeitintervall im Januar. 

\item Berechnung des Gezeitenvektors $g_{tid}$

Nun wird der zugehörige Gezeitenvektor $g_{tid}$ für den 20. Januar berechnet. Dabei wird das ganze auf Terme vom Grad 2 beschränkt. Die Berechnung des Gezeitenvektors lautet wie folgt: 

\begin{gather}
g_{tid} = \dfrac{\partial V^{tid}}{\partial r} e_r + \dfrac{1}{r} \dfrac{\partial V^{tid}}{\partial \phi} e_{\phi} + \dfrac{1}{r \cos \phi} \dfrac{\partial V^{tid}}{\partial \lambda} e_{\lambda}
\end{gather}

\begin{gather*}
= \begin{bmatrix}
-0.4434 ~\dfrac{m}{s^2} &-0.1148 ~\dfrac{m^2}{s^2} &-0.9148 ~\dfrac{m^2}{s^2}
\end{bmatrix}
\end{gather*}
\end{enumerate}

\section{Gezeitenkatalog HW95}

Aus dem Gezeitenkatalog HW95, welches 12935 Partialtiden der Sonne, des Mondes und einiger Planeten enthält, soll das vom Mond erzeugte Gezeitenpotential $v_{tid}$, sowie der zugehörige Gezeitenvektor $g_{tid}$ für denselben Beobachtunspunkt und dieselben Zeitpunkte wie in Aufgabe 1 berechnet werden. Die Gezeitenpotentiale für die Werte des Katalogs werden analog zu Aufgabe 1 berechnet und lauten: 

\begin{gather}
V_{HW}^{tid} = \begin{bmatrix}
-1.4625 &-1.3588& -1.2480&-1.1389&-1.0406
\end{bmatrix}~[m^2/s^2]
\end{gather}

Diese Ergebnisse beziehen sich, wie Aufgabe 1b), auf die Tage vom ersten bis zum fünften Januar. Die Differenzen zu den Ergebnissen aus Aufgabe 1b) ergeben sich als: 

\begin{gather*}
\Delta V^{tid} = \begin{bmatrix}
0.0062 & 0.0048 & 0.0025 & 0.0011 & 0.0018
\end{bmatrix}~[m^2/s^2]
\end{gather*} 

Nun wird der Gezeitenvektor für den 20. Januar analog zu Aufgabe 1c) berechnet. Dieser lautet: 

\begin{gather*}
g_{HW}^{tid} = 10^{-6} \cdot \begin{bmatrix}
-0.4440 ~\dfrac{m}{s^2} & -0.1166 ~\dfrac{m^2}{s^2} &  -0.9207 ~\dfrac{m^2}{s^2}
\end{bmatrix} 
\end{gather*}

Die Differenzen zu Aufgabe 1c) sehen wie folgt aus: 

\begin{gather*}
\Delta g_{tid} = 10^{-8} \cdot \begin{bmatrix}
0.0560 ~\dfrac{m}{s^2} &  0.1842 ~\dfrac{m^2}{s^2} &  0.5813 ~\dfrac{m^2}{s^2}
\end{bmatrix}
\end{gather*}

Fazit: Die Differenzen zwischen den Ergebnissen von Aufgabe 1 und denen dieser Aufgabe fallen relativ klein aus. Besonders die des Gezeitenvektors sind sehr klein, wobei hinsichtlich der Größenordnung des Vektors sie durchaus nachvollziehbar sind. 

