\chapter{Ausarbeitung}
\section{Aufgabe 1: Legendre-Funktionen}
\subsection{a}
Die unnormierten Legendrepolynome
\begin{gather}
	P_3(t) = \frac{2n-1}{n} t P_2(t) - \frac{n-1}{n}P_1(t) = \frac{5}{2}t^3-\frac{3}{2}t \\
	P_4(t) = \frac{2n-1}{n} t P_3(t) - \frac{n-1}{n}P_2(t) = \frac{35}{8}t^4 - \frac{15}{4}t^2 + \frac{3}{8}
\end{gather}
Normierungsfaktoren:
\begin{equation}
	N_{l,0} = \sqrt{(2l+1)}	
\end{equation}
$N_{0,0} = 1$, $N_{1,0} = \sqrt{3}$, $N_{2,0} = \sqrt{5}$, $N_{3,0} = \sqrt{7}$, $N_{4,0} = 3$.
\subsection{b}

\clearpage
\section{Aufgabe 2: Sphärisch-Harmonische Analyse}
\begin{equation}
	Y_{n,m}(t) = P_{n,m}(t) \begin{Bmatrix}
	\sin(m\lambda) \\
	\cos(m \lambda)
	\end{Bmatrix}
\end{equation}
\subsection{a}
Für $f_1$, weil $f_1$ nur mit $\theta$ abhängig ist und $\int_{0}^{2\pi}\sin(m\lambda) = 0$, nimmt man hier nur $\cos(m\lambda)$
\begin{equation}
	f_1(\lambda,\theta) = \cos^2\theta \Longrightarrow f_1(\lambda,t) = t^2 
\end{equation}
\begin{align}
	c_{n,m}^{f_1} = & \frac{1}{4\pi} \int_{-1}^{1} \int_{0}^{2\pi} N_{n.m} \cdot P_{n,m}(t) \cdot \cos(m\lambda) \cdot f_1(\lambda,t) d\lambda dt \\
	= & \frac{1}{4\pi} \int_{-1}^{1} N_{n.m} \cdot P_{n,m}(t) \cdot t^2  \left[\int_{0}^{2\pi} \cos(m\lambda) d\lambda\right]  dt
\end{align}
\begin{equation}
	\int_{0}^{2\pi} \cos(m\lambda) d\lambda = \begin{cases}
	0 & m \neq 0 \\
	2\pi & m=0
	\end{cases}
\end{equation}
Wenn $m=0$:
\begin{equation}
	c_{n,0}^{f_1} = \frac{N_{n,0}}{2} \int_{-1}^{1} P_{n,0}(t) \cdot t^2 dt
\end{equation}
Wenn $n$ ungerade ist, ist Grad von t von $P_{n,0}\cdot t^2$ auch ungerade. Weil $\int_{-1}^{1} t^{2n-1}dt = 0$, muss man nur $c_{0,0}^{f_1}$, $c_{2,0}^{f_1}$, $c_{4,0}^{f_1}$ berechnen.
\begin{align}
	& c_{0,0}^{f_1} = \frac{1}{2} \int_{-1}^{1} t^2 dt = \frac{1}{3} \\
	& c_{2,0}^{f_1} = \frac{\sqrt{5}}{2} \int_{-1}^{1} (\frac{3}{2}t^2-\frac{1}{2})t^2dt = \frac{2\sqrt{5}}{15} \\
	& c_{4,0}^{f_1} = \frac{3}{2} \int_{-1}^{1} (\frac{35}{8}t^4 - \frac{15}{4}t^2 + \frac{3}{8})t^2dt = 0
\end{align}
Für $f_2$, weil:
\begin{align}
	& \int_{0}^{2\pi}\cos(kt)\sin(nt)dt = 0 \\
	& \int_{0}^{2\pi}\sin(kt)\sin(nt)dt = \begin{Bmatrix}
	\pi & \text{für  } k=n \\
	0 &\text{sonst}
	\end{Bmatrix}
\end{align}
Man muss nur für $m=1$ für $Y_{n,1} = P_{n,1}\sin(\lambda)$ berechnen. Wenn $n$ gerade ist, ist Grad von t ungerade. Deshalb sind die Berechnung nur für $c_{1,1}^{f_2}$ und $c_{3,1}^{f_2}$ notwendig.
\begin{align}
	c_{1,1}^{f_1} = \ & \frac{1}{4\pi}\int_{-1}^{1}(\sqrt{1-t^2})^3\sqrt{3}\sqrt{1-t^2}\pi dt\\
	= \ & \frac{\sqrt{3}}{4} \int_{-1}^{1} (t^4 - 2t^2 +1)dt \\
	= \ & \frac{4\sqrt{3}}{15} \\
	c_{3,1}^{f_2} = \ & \frac{1}{4\pi}\int_{-1}^{1}(\sqrt{1-t^2})^3\sqrt{\frac{7}{6}}\cdot \frac{3}{2}(5t^2-1)\sqrt{1-t^2} \pi dt \\
	= \ & \frac{3}{8}\sqrt{\frac{7}{6}}\int_{-1}^{1}(5t^6 - 11t^4 + 7t^2-1) dt \\
	= \ & -\frac{4}{35} \sqrt{\frac{7}{6}} \\ 
	= \ & -\frac{2\sqrt{42}}{105}
\end{align}
Für $f_3$:\\
Ähnlich wie für $f_1$, man nimmt da nur $\cos(m\lambda)$ und $c_{n,m}^{f3} = 0$ wenn $m \neq 0$. Man benennt:
\begin{equation*}
	J_k = \int_{-1}^{1} t^k \cosh (t) dt
\end{equation*}
Dann:
\begin{align}
	c_{0,0}^{f_3} = \ & \frac{1}{2} \int_{-1}^{1} 1 \cosh (t) dt = \frac{1}{2} J_0 \\
	c_{1,0}^{f_3} = \ & \frac{\sqrt{3}}{2} \int_{-1}^{1} t \cosh (t) dt = \frac{\sqrt{3}}{2}J_1 \\
	c_{2,0}^{f_3} = \ & \frac{\sqrt{5}}{2} \int_{-1}^{1} (\frac{3}{2}t^2-\frac{1}{2})\cosh (t) dt = \frac{3\sqrt{5}}{4}J_2-\frac{\sqrt{5}}{4} J_0 \\
	c_{3,0}^{f_3} = \ & \frac{\sqrt{7}}{2} \int_{-1}^{1} (\frac{5}{2}t^3 - \frac{3}{2}t)\cosh (t) dt = \frac{5\sqrt{7}}{4}J_3 - \frac{3\sqrt{7}}{4} J_1 \\
	c_{4,0}^{f_3} = \ & \frac{3}{2} \int_{-1}^{1} (\frac{35}{8}t^4 - \frac{15}{4}t^2 + \frac{3}{8}) \cosh (t) dt = \frac{105}{16}J_4 - \frac{45}{8}J_2 + \frac{9}{16}J_0 \\
\end{align}
$J_0$ kann direkt berechnet werden:
\begin{equation}
	J_0 = \int_{-1}^{1} \cosh (t)dt = \left[\sinh(t)\right]^1_{-1} = 2\sinh(1)
\end{equation}
$J_1$ kann via partielle Integration berechnet werden:
\begin{align}
	J_1 & = \int_{-1}^{1} t \cosh(t) dt \\
	 	& = \left[t\sinh(t)\right]^{1}_{-1} - \int_{-1}^{1} \sinh(t) dt \\
	 	& = \left[t\sinh(t)\right]^{1}_{-1} - \left[\cosh(t)\right]^{1}_{-1} \\
	 	& = 0
\end{align}
$J_2$ bis $J_4$ werden via partielle Integration rekursive berechnet: 
\begin{align}
	J_k = & \int_{-1}^{1} t^k \cosh(t) dt \\
		= & \left[t^k\sinh(t)\right]_{-1}^{1} - \int_{-1}^{1}k t^{k-1} \sinh(t) dt \\
		= & \left[t^k\sinh(t)\right]_{-1}^{1} - k\left[t^{k-1}\cosh(t)\right]_{-1}^{1} + k(k-1)\int_{-1}^{1} \cosh(t)t^{k-2}dt \\
		= & \left[t^k\sinh(t)\right]_{-1}^{1} - k\left[t^{k-1}\cosh(t)\right]_{-1}^{1} + k(k-1)J_{k-2}
\end{align}
Wenn $t$ ungerade ist: $J_k = 0-0+k(k-1)J_{k-2}$. Denn $J_1 = 0$, $J_k = 0$ für alle ungerade $k$. \\\\
Für gerade $k$:
\begin{align}
	J_k = 2\sinh(1) - 2k\cosh(1) + k(k-1)J_{k-2} 
\end{align}
Ansatz:
\begin{align}
	J_2 = \ & 2\sinh(1) - 4\cosh(1) + 2J_0 \\
	= \ & 2\sinh(1) - 4\cosh(1) + 4 \sinh(1) \\
	= \ & 6\sinh(1) - 4\cosh(1) \\
	J_4 = \ & 2\sinh(1) - 8\cosh(1) + 12 J_2\\
		= \ & 2\sinh(1) - 8\cosh(1) + 12 (6\sinh(1) - 4\cosh(1))\\
		= \ & 74 \sinh(1) - 56\cosh(1)
\end{align}
Berechnung von Koeffizienten:
\begin{align}
	c_{0,0}^{f_3} = \ & \sinh(1) \\
	c_{1,0}^{f_3} = \ & 0 \\
	c_{2,0}^{f_3} = \ & 4\sqrt{5}\sinh(1) - 3\sqrt{5} \cosh(1) \\
	c_{3,0}^{f_3} = \ & 0 \\
	c_{4,0}^{f_3} = \ & 453\sinh(1) - 345\cosh(1)
\end{align}
\subsection{b}
Für $f_1$: 
\begin{align}
	\sum_n \sum_m c_{n,m}^{f_1} \bar{Y}_{n,m}(\lambda,t)  = \ & c_{0,0}^{f_1} \bar{Y}_{0,0}(\lambda,t) + c_{2,0}^{f_1} \bar{Y}_{2,0}(\lambda,t) +c_{4,0}^{f_1} \bar{Y}_{4,0}(\lambda,t) \\
	   = \ & \frac{1}{3} \cdot 1 \cdot \cos(0) + \frac{2\sqrt{5}}{15} \cdot \sqrt{5} \cdot (\frac{3}{2}t^2 - \frac{1}{2}) \cdot \cos(0) + 0 \\
	   = \ & t^2 \\
	   = \ & \cos^2(\theta)
\end{align}
Für $f_2$:
\begin{align}
	\sum_n \sum_m c_{n,m}^{f_2} \bar{Y}_{n,m}(\lambda,t) = \ & c_{1,1}^{f_2} \bar{Y}_{1,1}(\lambda,t) + c_{3,1}^{f_2} \bar{Y}_{3,1}(\lambda,t) \\
	= \ & \frac{4\sqrt{3}}{15} \cdot  \sqrt{3} \sqrt{1-t^2} \sin(\lambda) - \frac{2\sqrt{42}}{105} \cdot \sqrt{\frac{7}{6}}\cdot \frac{3}{2}(5t^2-1)\sqrt{1-t^2} \sin(\lambda) \\
	= \ & \frac{3}{4} \sin \theta \sin \lambda - \frac{1}{4} \sin3\theta \sin \lambda
\end{align}
Mit MATLAB ist das maximale Ergebnis von $\frac{3}{4} \sin \theta \sin \lambda - \frac{1}{4} \sin3\theta \sin \lambda - \sin^3(\theta)\sin(\lambda)$ im Bereich von $10^{-16}$ (bis zum $n = m = 4$)\\\\
Für $f_3$:
\begin{align}
	\sum_n \sum_m c_{n,m}^{f_2} \bar{Y}_{n,m}(\lambda,t) = \ & c_{0,0}^{f_3} \bar{Y}_{0,0}(\lambda,t) + c_{2,0}^{f_3} \bar{Y}_{2,0}(\lambda,t) + c_{4,0}^{f_3} \bar{Y}_{4,0}(\lambda,t) \\
	= \ & \sinh(1) + (4\sqrt{5}\sinh(1)-3\sqrt{5}\cosh(1)) \cdot \sqrt{5} \cdot \frac{1}{4}(1+3\cos(2\theta)) \\
	& + (453\sinh(1) - 345\cosh(1)) \cdot 3 \cdot (\frac{35}{8}\cos^4\theta - \frac{15}{4}\cos^2\theta + \frac{3}{8})
\end{align}
Mit MATLAB ist das maximale Ergebnis von $\cosh(\cos \theta) -\sum_n \sum_m c_{n,m}^{f_2} \bar{Y}_{n,m}(\lambda,t) $ im Bereich von $10^{-5}$ (bis zum $n = m = 4$)
\clearpage
\section{Aufgabe 3: Rotation von Kugelflächenfunktionen}
